\documentclass[xcolor={dvipsnames},aspectratio=169,10pt]{beamer}

\input{preamble.tex}

\title{Obtención de los Coeficientes de la forma canónica para la Elipse, hipérbolas y parábolas}
\subtitle{Determinación analítica de las cónicas}
\author{Grupo 12}
\date{December 05, 2023}
\titlegraphic{
  \begin{tikzpicture}[overlay, remember picture]
    \node[%
      above right=0.35cm and -0.2cm of current page footer area.south west,
      anchor=south west,
      inner sep=0pt] {%
      \usebeamerfont{footline}
    };
    % \node[%
    %   above left=0.35cm and 0cm of current page footer area.south east,
    %   anchor=south east,
    %   inner sep=0pt]{\qrcode[height=1.5cm]{https://github.com/axvg/presentaciones}};
  \end{tikzpicture}
}

\begin{document}

\maketitle%

\begin{frame}{Contenidos}
  \setbeamertemplate{section in toc}[sections numbered]
  \tableofcontents[hideallsubsections]
\end{frame}

\section{Introduccion}

\begin{frame}{Formas cuadraticas}
    \frametitle{Formas cuadraticas}
    \begin{definition}
      Una forma cuadratica en n variables $x_{1}, x_{2}, . . . , x_{n}$ es una combinacion lineal de los
      productos $x_{i} x_{j}$, esto es, una combinacion lineal de cuadrados $x_{1}^2 , x_{2}^2 , . . . , x_{n}^2$ y
      terminos $x_{1}x_{2}, x_{1}x_{3}, . . . , x_{1}x_{n}, x_{2}x_{3}, . . . , x_{2}x_{n}, . . . , x_{n-1}x_{n}$
    \end{definition}

  \begin{example}
    \begin{itemize}
        \item $q = x^2 - y^2 + 4xy$ and $q = x^2 + 3y^2 - 2xy$ are quadratic forms in $x$ and $y$.
        \item $q = -4x_{21} + x_{22}^2 + 4x_{23} + 6x_{1}x_{3}$ is a quadratic form in $x_{1}, x_{2}$ and $x_{3}$.
        \item The most general quadratic form in $x_{1}, x_{2}, x_{3}$ is $a_{1}x_{21} + a_{2}x_{22}^2 + a_{3}x_{23} + a_{12}x_{1}x_{2} + a_{13}x_{1}x_{3} + a_{23}x_{2}x_{3}$.
    \end{itemize}
  \end{example}
\end{frame}

\begin{frame}[fragile]{Formas cuadraticas}
    \frametitle{A}
    \lipsum[1-1]
\end{frame}

\begin{frame}{2}
    \frametitle{A}
    \lipsum[1-1]
\end{frame}

\begin{frame}{3}
    \frametitle{A}
    \lipsum[1-1]
\end{frame}

\section{Seccion 2}

\begin{frame}[fragile]{2.1}
    \frametitle{A}
    \lipsum[1-1]
\end{frame}

\begin{frame}[fragile]{2.2}
    \frametitle{A}
    \lipsum[1-1]
\end{frame}


\begin{frame}[fragile]{2.3}
    \frametitle{A}
    \lipsum[1-1]
\end{frame}

\begin{frame}[fragile]{2.4}
    \frametitle{A}
    \lipsum[1-1]
\end{frame}

\section{Seccion3}

\begin{frame}[fragile]{IX}
    \frametitle{A}
    \lipsum[1-1]
\end{frame}

% \begin{noindent}
\begin{frame}[fragile]{X}
    \frametitle{A}
    \lipsum[1-1]
\end{frame}
% \end{noindent}

\begin{frame}{Bibliografia}
  \begin{itemize}
    \item Applications of Linear Algebra in Various Fields (Part-1): \url{https://www.researchgate.net/publication/356818396_Applications_of_Linear_Algebra_in_Various_Fields_Part-1}
    \item Álgebra lineal y geometría cartesiana - Juan de Burgos Román
    \item Practical Linear Algebra: A Geometry Toolbox - Gerald Farin, Dianne Hansford
  \end{itemize}
\end{frame}

\begin{frame}[standout]
  Gracias \\
\end{frame}

\end{document}