\documentclass[xcolor={dvipsnames},aspectratio=169,10pt]{beamer}

% utility packages
\usepackage{multicol}
\usepackage{relsize}
\usepackage{amsthm}
\usepackage[spanish]{babel}
\usepackage{biblatex}
\usepackage{fontawesome}
\usepackage{pgfplots}
\usepackage{enumitem}
\usepackage{empheq}
\usepackage{xcolor}
\usepackage{epigraph}

% better text justifying
\usepackage{microtype}
% justify text inside list environment
% Ref: http://liam0205.me/2017/04/11/justifying-in-beamer-s-lists/
\usepackage{ragged2e}
\makeatletter
\patchcmd{\itemize}{\raggedright}{\justifying}{}{}
\patchcmd{\beamer@enum@}{\raggedright}{\justifying}{}{}
\patchcmd{\@@description}{\raggedright}{\justifying}{}{}
\makeatother

% math related packages
\usepackage{amsmath}
\usepackage[ruled,vlined]{algorithm2e}
\SetAlCapNameFnt{\scriptsize}
\SetAlCapFnt{\scriptsize}
\SetAlFnt{\scriptsize}

% figure related packages
\usepackage{graphicx}
\usepackage[scale=2]{ccicons}
\usepackage{tikz}
\usepackage{tikzpagenodes}
\usetikzlibrary{decorations.pathreplacing}
\usetikzlibrary{positioning}

% table related packages
\usepackage{array}
\usepackage{booktabs}
\usepackage{multirow}
\usepackage{colortbl}
\newcommand{\tabincell}[2]{\begin{tabular}{@{}#1@{}}#2\end{tabular}}

% hyperref setting
\hypersetup{
  unicode,
  psdextra,
  bookmarksnumbered=true,
  bookmarksopen=true,
  bookmarksopenlevel=3,
  bookmarksdepth=4,
  pdfcenterwindow=true,
  pdfstartview={Fit},
  pdfpagemode={FullScreen},
  pdfpagelayout={SinglePage},
}
\usepackage{bookmark}

% beamer theme
\usetheme{metropolis}
\metroset{block=fill,numbering=fraction}

% caption style
\usepackage{subcaption}
\setlength\abovecaptionskip{3pt}
\setbeamerfont{caption}{size=\scriptsize}
\renewcommand{\figurename}{Fig.}
\captionsetup{labelformat=empty,labelsep=none,textfont={bf,it}}

% Ref: https://github.com/gpoore/minted/blob/master/source/minted.dtx
\newenvironment{latexexample}
{\VerbatimEnvironment\begin{VerbatimOut}[gobble=3]{example.out}}{\end{VerbatimOut}%
  \begin{center}
    \begin{minipage}{0.47\linewidth}%
      \inputminted[resetmargins,fontsize=\scriptsize]{latex}{example.out}%
    \end{minipage}%
    \hspace{0.05\linewidth}%
    \begin{minipage}{0.47\linewidth}%
      \begin{framed}
        \setlength{\parindent}{2em}%
        \input{example.out}%
      \end{framed}
    \end{minipage}%
  \end{center}
}

\newenvironment{mathexample}
{\VerbatimEnvironment\begin{VerbatimOut}[gobble=3]{example.out}}{\end{VerbatimOut}%
  \begin{center}
    \begin{minipage}{0.47\linewidth}%
      \inputminted[resetmargins,fontsize=\scriptsize]{latex}{example.out}%
    \end{minipage}%
    \hspace{0.05\linewidth}%
    \begin{minipage}{0.47\linewidth}%
      \begin{framed}
        \[ \input{example.out} \]
      \end{framed}
    \end{minipage}%
  \end{center}
}

\newenvironment{mathexamples}
{\VerbatimEnvironment\begin{VerbatimOut}[gobble=3]{example.out}}{\end{VerbatimOut}%
  \begin{center}
    \begin{minipage}{0.47\linewidth}%
      \inputminted[resetmargins,fontsize=\scriptsize]{latex}{example.out}%
    \end{minipage}%
    \hspace{0.05\linewidth}%
    \begin{minipage}{0.47\linewidth}%
      \begin{framed}
        \directlua{
          local first = true
          for line in io.lines('example.out') do
          if first then
          first = false
          else
          tex.print('\\newline ')
          end
          tex.print('$' .. line .. '$')
          end
        }
      \end{framed}
    \end{minipage}%
  \end{center}
}

\title{Obtención de los Coeficientes de la forma canónica para la Elipse, hipérbolas y parábolas}
\subtitle{Determinación analítica de las cónicas}
\author{Grupo 12}
\date{December 05, 2023}
\titlegraphic{
  \begin{tikzpicture}[overlay, remember picture]
    \node[%
      above right=0.35cm and -0.2cm of current page footer area.south west,
      anchor=south west,
      inner sep=0pt] {%
      \usebeamerfont{footline}
    };
    % \node[%
    %   above left=0.35cm and 0cm of current page footer area.south east,
    %   anchor=south east,
    %   inner sep=0pt]{\qrcode[height=1.5cm]{https://github.com/axvg/presentaciones}};
  \end{tikzpicture}
}

\begin{document}

\maketitle%

\begin{frame}{Contenidos}
  \setbeamertemplate{section in toc}[sections numbered]
  \tableofcontents[hideallsubsections]
\end{frame}

\section{Introduccion}

\begin{frame}{Formas cuadraticas}
    \frametitle{Formas cuadraticas}
    \begin{definition}
      Una forma cuadratica en n variables $x_{1}, x_{2}, . . . , x_{n}$ es una combinacion lineal de los
      productos $x_{i} x_{j}$, esto es, una combinacion lineal de cuadrados $x_{1}^2 , x_{2}^2 , . . . , x_{n}^2$ y
      terminos $x_{1}x_{2}, x_{1}x_{3}, . . . , x_{1}x_{n}, x_{2}x_{3}, . . . , x_{2}x_{n}, . . . , x_{n-1}x_{n}$
    \end{definition}

  \begin{example}
    \begin{itemize}
        \item $q = x^2 - y^2 + 4xy$ and $q = x^2 + 3y^2 - 2xy$ son formas cuadraticas en $x$ y $y$.
        \item $q = -4x_{21} + x_{22}^2 + 4x_{23} + 6x_{1}x_{3}$ es una forma cuadratica en $x_{1}, x_{2}$ y $x_{3}$.
        \item La formas cuadratica general de $x_{1}, x_{2}, x_{3}$ es $a_{1}x_{21} + a_{2}x_{22}^2 + a_{3}x_{23} + a_{12}x_{1}x_{2} + a_{13}x_{1}x_{3} + a_{23}x_{2}x_{3}$.
    \end{itemize}
  \end{example}
\end{frame}

\begin{frame}{Formas cuadraticas}
  \frametitle{Formas cuadraticas}
  Las formas cuadraticas pueden ser escritas de la forma matricial $q(x) = x^{T}Ax$ donde $A$ es una matriz simetrica $n 
  \times n$ y $x$ es un vector columna $n \times 1$.

  La matriz $A$ es llamada la matriz de la forma cuadratica $q$.

  % Suppose q = x2
  % 1 − x2
  % 2 + 4x1x2. The coefficients of x2
  % 1 and x2
  % 2 are 1 and −1,
  % respectively, so we put these, in order, into the two diagonal positions of a
  % matrix A. The coefficient of x1x2 is 4, which we split equally between the
  % (1, 2) and (2, 1) positions, putting a 2 in each place.
  % traduce lo de arriba y ponlo como ejemplo
  \begin{example}
    Supongamos que $q = x_1^2 - x_2^2 + 4x_1x_2$. Los coeficientes de $x_1^2$ y $x_2^2$ son 1 y -1, 
    respectivamente, por lo que colocamos estos, en orden, en las dos posiciones diagonales de una matriz A. 
    El coeficiente de $x_1x_2$ es 4, que dividimos equitativamente entre las posiciones (1, 2) y (2, 1), 
    colocando un 2 en cada lugar.
  \end{example}
\end{frame}

\begin{frame}{Formas cuadraticas}
  \frametitle{Formas cuadraticas}
  Asi tenemos que:
  \begin{equation*}
    A = \begin{bmatrix}
      1 & 2 \\
      2 & -1
    \end{bmatrix}
  \end{equation*}
  y x = $\begin{bmatrix}
    x_1 \\
    x_2
  \end{bmatrix}$
  Luego:
  \begin{equation*}
    q(x) = x^{T}Ax = \begin{bmatrix}
      x_1 & x_2
    \end{bmatrix}
    \begin{bmatrix}
      1 & 2 \\
      2 & -1
    \end{bmatrix}
    \begin{bmatrix}
      x_1 \\
      x_2
    \end{bmatrix}
    = x_1^2 - x_2^2 + 4x_1x_2
  \end{equation*}
\end{frame}

\begin{frame}{3}
    \frametitle{A}
    \lipsum[1-1]
\end{frame}

\section{Seccion 2}

\begin{frame}[fragile]{2.1}
    \frametitle{A}
    \lipsum[1-1]
\end{frame}

\begin{frame}[fragile]{2.2}
    \frametitle{A}
    \lipsum[1-1]
\end{frame}


\begin{frame}[fragile]{2.3}
    \frametitle{A}
    \lipsum[1-1]
\end{frame}

\begin{frame}[fragile]{2.4}
    \frametitle{A}
    \lipsum[1-1]
\end{frame}

\section{Seccion3}

\begin{frame}[fragile]{IX}
    \frametitle{A}
    \lipsum[1-1]
\end{frame}

% \begin{noindent}
\begin{frame}[fragile]{X}
    \frametitle{A}
    \lipsum[1-1]
\end{frame}
% \end{noindent}

\begin{frame}{Bibliografia}
  \begin{itemize}
    \item Applications of Linear Algebra in Various Fields (Part-1): \url{https://www.researchgate.net/publication/356818396_Applications_of_Linear_Algebra_in_Various_Fields_Part-1}
    \item Álgebra lineal y geometría cartesiana - Juan de Burgos Román
    \item Practical Linear Algebra: A Geometry Toolbox - Gerald Farin, Dianne Hansford
  \end{itemize}
\end{frame}

\begin{frame}[standout]
  Gracias \\
\end{frame}

\end{document}