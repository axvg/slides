\documentclass{beamer}

\usepackage{lipsum}

\title{Obtención de los Coeficientes de la forma canónica para la Elipse, hipérbolas y parábolas. \newline
Determinación analítica de las cónicas}
\author{Grupo 12}
\date{\today}

\usetheme{Copenhagen}

\begin{document}
\maketitle

\begin{frame}{Contenido}
    \setbeamertemplate{section in toc}[sections numbered]
    \tableofcontents[hideallsubsections]
\end{frame}

\section{Introducción}
\begin{frame}
    \frametitle{Descripción}
    Este proyecto se dedica exclusivamente a la "Definición de
    Cónica y su ecuación general interpretación geométrica.".....
\end{frame}

\begin{frame}
    \frametitle{Conceptos}
    \lipsum[1-1]
\end{frame}

\section{Aplicaciones}
\begin{frame}
    \frametitle{Aplicaciones}
    \lipsum[1-1]
\end{frame}

\section{Ejercicio}
\begin{frame}
    \frametitle{Ejercicio}
    \lipsum[1-1]
\end{frame}

\section{Bibliografia}
\begin{frame}{Bibliografia}
\begin{itemize}
    \item Libro~\LaTeX: \url{https://en.wikibooks.org/wiki/LaTeX}
    \item Libro 2
    \item Libro 3
    \item Libro 4
    \item \alert{Google}
\end{itemize}
\end{frame}
  
\end{document}